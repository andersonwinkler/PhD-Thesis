\chapter{Introduction}
\setstretch{\lspac}

The previous chapter discussed how permutation methods, for being free of various assumptions related to classical parametric tests, could better adapt to the growing variety of experimental imaging methods. In this chapter, these ideas are further extended to non-parametrically allow \emph{joint inference} on more than one modality, that is, how to infer about hypotheses concerning these modalities when more than one image is available for each subject. Examples of such modalities include the same cited in Chapter~\ref{sec:perm:permglm}, such as, but not limited to, positron emission tomography (\textsc{pet}), functional magnetic resonance imaging (\textsc{fmri}), tensor-based morphometry (\textsc{tbm}), diffusion tensor imaging (\textsc{dti}), cortical thickness and surface area, cerebral perfusion, as well as various others.

Further to these examples, it is also the case that the same imaging modality is often subdivided so as to better characterise certain physical properties --- including morphology and function --- of the biological tissue. As an example, diffusion-weighted images are often used to generate maps of fractional anisotropy (\textsc{fa}), mean diffusivity (\textsc{md}), radial diffusivity (\textsc{rd}), as well as lengths of the eigenvectors of the diffusion tensor and other measurements. Another example is the use of independent component analysis (\textsc{ica}) to decompose \textsc{fmri} time series into a set of timecourses and spatial maps. Only some of these components might be of actual interest, or the effect of interest might be split into more than one; in either case, a strategy that could combine the information from these into a single inferential step tends to be more meaningful than various separate tests. \citep{Fisher1932}